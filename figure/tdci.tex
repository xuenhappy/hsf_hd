\documentclass[tikz,border=10pt]{standalone}
\usepackage{tikz}
\usepackage{amsmath}
\usepackage{amssymb}
\usepackage{ctex} % 中文支持

\begin{document}
% 加载必要的 TikZ 库
\usetikzlibrary{shapes.geometric, arrows.meta, positioning, calc, shadows, backgrounds, fit}

% --- 颜色定义 (HSF-HD 风格) ---
\definecolor{hsfBlue}{RGB}{22, 160, 133}   % 几何/结构
\definecolor{hsfRed}{RGB}{192, 57, 43}     % 外部/激波
\definecolor{hsfPurple}{RGB}{142, 68, 173} % 价值/目的
\definecolor{hsfOrange}{RGB}{211, 84, 0}   % 意志/控制
\definecolor{hsfDark}{RGB}{44, 62, 80}     % 核心思维



\begin{tikzpicture}[
    node distance=2.5cm and 4cm,
    % --- 节点样式 ---
    core/.style={
        circle, 
        draw=hsfDark, 
        ultra thick, 
        fill=hsfDark!10, 
        minimum size=3.5cm, 
        align=center, 
        drop shadow,
        font=\bfseries
    },
    component/.style={
        rectangle, 
        rounded corners=8pt, 
        draw=black!60, 
        thick, 
        minimum width=4cm, 
        minimum height=1.8cm, 
        align=center, 
        drop shadow,
        font=\small
    },
    % --- 连线样式 ---
    link/.style={
        <->, 
        >={Latex[length=3mm, width=2mm]}, 
        line width=1.5pt, 
        draw=gray!80
    },
    dashed_link/.style={
        ->, 
        >={Latex[length=2mm, width=1.5mm]}, 
        line width=1pt, 
        draw=gray!60,
        dashed
    },
    % --- 标签样式 ---
    label_text/.style={
        fill=white, 
        fill opacity=0.9, 
        text opacity=1, 
        font=\footnotesize\sffamily, 
        align=center,
        inner sep=2pt,
        text=black!80
    }
]

    % =================================================
    % 1. 核心节点:TDE (思维流)
    % =================================================
    \node[core] (TDE) {
        \textcolor{hsfDark}{\Large \textbf{TDE}}\\
        \textbf{目的论狄拉克方程}\\
        \textit{思维流 $\Psi$}\\
        (推理/演化)
    };

    % =================================================
    % 2. 周围节点:内部组件
    % =================================================
    
    % 左侧:CEFE (结构)
    \node[component, left=of TDE, fill=hsfBlue!10, draw=hsfBlue] (CEFE) {
        \textcolor{hsfBlue}{\large \textbf{CEFE}}\\
        \textbf{认知爱因斯坦方程}\\
        \textit{度量 $g_{\mu\nu}$}\\
        (记忆/习惯/形)
    };

    % 右侧:CYME (价值)
    \node[component, right=of TDE, fill=hsfPurple!10, draw=hsfPurple] (CYME) {
        \textcolor{hsfPurple}{\large \textbf{CYME}}\\
        \textbf{认知杨-米尔斯方程}\\
        \textit{规范势 $\mathcal{A}_\mu$}\\
        (偏好/目的/质)
    };

    % 下方:TCE (意志)
    \node[component, below=of TDE, fill=hsfOrange!10, draw=hsfOrange, yshift=-0.5cm] (TCE) {
        \textcolor{hsfOrange}{\large \textbf{TCE}}\\
        \textbf{目的论控制方程}\\
        \textit{宏观算子 $\hat{\mathcal{O}}$}\\
        (控制/注意/元认知)
    };

    % =================================================
    % 3. 外部节点:环境
    % =================================================
    
    % 上方:ENV (环境)
    \node[component, above=of TDE, fill=hsfRed!10, draw=hsfRed, yshift=0.5cm] (ENV) {
        \textcolor{hsfRed}{\large \textbf{External Env}}\\
        \textbf{外部物理世界}\\
        \textit{源流 $\vec{J}_{ext}$ / 动作 Action}
    };

    % =================================================
    % 4. 绘制背景框 (区分内外)
    % =================================================
    \begin{scope}[on background layer]
        \node[fit=(CEFE)(CYME)(TCE)(TDE), 
              draw=gray!30, dashed, rounded corners=15pt, 
              fill=gray!5, inner sep=15pt,
              label={[anchor=north west, xshift=10pt, yshift=-5pt, text=gray]north west:\textbf{智能体内部 (Internal Manifold)}}] (box) {};
    \end{scope}

    % =================================================
    % 5. 绘制连线 (主要耦合) - 星型拓扑无交叉
    % =================================================

    % I. 现实耦合 (ENV <-> TDE)
    \draw[link, draw=hsfRed] (ENV) -- (TDE) 
        node[midway, label_text] {I. 现实耦合\\(感知 $\downarrow$ / 行动 $\uparrow$)};

    % II. 结构耦合 (TDE <-> CEFE)
    \draw[link, draw=hsfBlue] (TDE) -- (CEFE) 
        node[midway, label_text] {II. 结构耦合\\(认知应力 $\to$ / $\leftarrow$ 几何惯性)};

    % III. 价值耦合 (TDE <-> CYME)
    \draw[link, draw=hsfPurple] (TDE) -- (CYME) 
        node[midway, label_text] {III. 价值耦合\\(价值流 $\to$ / $\leftarrow$ 洛伦兹力)};

    % IV. 控制耦合 (TDE <-> TCE)
    \draw[link, draw=hsfOrange] (TDE) -- (TCE) 
        node[midway, label_text] {IV. 控制耦合\\(偏差检测 $\downarrow$ / $\uparrow$ 算子注入)};

    % =================================================
    % 6. 隐式/次级耦合 (可选,使用曲线避免交叉)
    % =================================================
    
    % TCE -> CEFE (塑性重构)
    \draw[dashed_link, bend left=45] (TCE.west) to node[label_text, font=\tiny] {V. 塑性重构} (CEFE.south);
    
    % TCE -> CYME (价值重估)
    \draw[dashed_link, bend right=45] (TCE.east) to node[label_text, font=\tiny] {VI. 价值重估} (CYME.south);

\end{tikzpicture}

\end{document}