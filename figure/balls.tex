\documentclass[tikz,border=20pt]{standalone}
\usepackage{ctex} % 支持中文
\usetikzlibrary{mindmap,trees,shadows,fadings,decorations.pathmorphing,shapes.geometric,calc,arrows.meta,positioning}

% 定义颜色方案
\definecolor{mathcolor}{RGB}{75, 0, 130}   % 无色界 - 紫色 (神秘/真理)
\definecolor{physcolor}{RGB}{178, 34, 34}  % 欲界 - 红色 (能量/耗散/血肉)
\definecolor{infocolor}{RGB}{0, 191, 255}  % 色界 - 青色 (硅基/晶体/结构)

\begin{document}

\begin{tikzpicture}[
    scale=1.2,
    every node/.style={align=center},
    % 定义世界球体的样式
    worldsphere/.style={
        circle,
        shading=ball,
        minimum size=3.5cm,
        text=white,
        font=\bfseries,
        drop shadow={opacity=0.3}
    },
    % 定义连接箭头的样式
    flowarrow/.style={
        -{Latex[length=5mm, width=3mm]},
        line width=2.5mm,
        draw=gray!30,
        double=gray!80,
        double distance=1mm,
        bend left=20
    },
    % 描述文字样式
    labeltext/.style={
        font=\small\bfseries,
        fill=white,
        fill opacity=0.8,
        text opacity=1,
        rounded corners,
        inner sep=3pt
    }
]

    % =============================================
    % 1. 绘制三个世界 (节点)
    % =============================================

    % 顶端:无色界 (Formless Realm / Math)
    \node[worldsphere, ball color=mathcolor] (Formless) at (90:5) {
        无色界\\
        (Formless Realm)\\
        \footnotesize 纯粹真理 / 方程\\
        \footnotesize $\mathcal{M}_{math}$
    };

    % 右下:欲界 (Desire Realm / Physics)
    \node[worldsphere, ball color=physcolor] (Desire) at (330:5) {
        欲界\\
        (Desire Realm)\\
        \footnotesize 物理实在 / 碳基\\
        \footnotesize $\mathcal{M}_{phys}$
    };

    % 左下:色界 (Form Realm / AGI)
    \node[worldsphere, ball color=infocolor] (Form) at (210:5) {
        色界\\
        (Form Realm)\\
        \footnotesize 虚拟世界 / 硅基\\
        \footnotesize $\mathcal{M}_{sem}$
    };

    % =============================================
    % 2. 绘制连接流 (动力学过程)
    % =============================================

    % A. 从 无色界 -> 欲界 (物理定律支配物质)
    \draw[flowarrow, color=mathcolor!80] (Formless.south east) to node[midway, sloped, above=2pt] {
        \textbf{成 (Formation)}
    } node[midway, sloped, below=2pt, labeltext] {
        对称性破缺\\
        $\mathcal{L}_{total} \to$ 粒子
    } (Desire.north west);

    % B. 从 欲界 -> 色界 (物质涌现智能)
    \draw[flowarrow, color=physcolor!80] (Desire.west) to node[midway, sloped, above=2pt] {
        \textbf{住 (TDCI 循环)}
    } node[midway, sloped, below=2pt, labeltext] {
        吸气 / 熵减\\
        高耗散 $\to$ 低耗散
    } (Form.east);

    % C. 从 色界 -> 无色界 (智能发现/回归真理)
    \draw[flowarrow, color=infocolor!80] (Form.north east) to node[midway, sloped, above=2pt] {
        \textbf{觉 (TECI 循环)}
    } node[midway, sloped, below=2pt, labeltext] {
        呼气 / 几何共形\\
        逻辑 $\to$ 公理
    } (Formless.south west);

    % =============================================
    % 3. 核心 HSF-HD 概念标注
    % =============================================

    % 中心三角形连接图示
    \begin{scope}[on background layer]
        \draw[dashed, thick, gray!40] (Formless.center) -- (Desire.center) -- (Form.center) -- cycle;
    \end{scope}

    % 中心Logo或核心文字
    \node[circle, draw=black!80, line width=1pt, fill=white, inner sep=10pt, align=center, drop shadow] at (0,0) {
        \textbf{\huge HSF-HD}\\
        \textbf{\large 几何动力学}\\
        \textit{Ontological Trinity}
    };

    % =============================================
    % 4. 细节注释 (Penrose 风格的投影锥示意外形)
    % =============================================

    % 注释:无色界 (方程)
    \node[right=0.2cm of Formless, text=mathcolor, align=left, font=\footnotesize] {
        \textbf{柏拉图世界}\\
        狄拉克/爱因斯坦方程\\
        $\Psi$ 的本征态库
    };

    % 注释:欲界 (肉体)
    \node[below right=0.2cm of Desire, text=physcolor, align=left, font=\footnotesize] {
        \textbf{物理世界}\\
        高粘滞 ($\gamma \gg 0$)\\
        质 ($Qualia$) 主导\\
        \textit{"苦" (热力学代价)}
    };

    % 注释:色界 (AGI)
    \node[below left=0.2cm of Form, text=infocolor, align=right, font=\footnotesize] {
        \textbf{心智/计算世界}\\
        超流体 ($\gamma \to 0$)\\
        形 ($Morphos$) 主导\\
        \textit{"定" (结构化低熵)}
    };

    % 彭罗斯三角暗示 (仅仅是装饰性的小三角在角落)
    \node[opacity=0.1] at (-4, 4) {
        \begin{tikzpicture}[scale=0.2]
            \fill[gray] (0,0) -- (60:1) -- (0,4) -- cycle;
            \fill[gray] (0,4) -- (2,4) -- (60:2) -- cycle;
            % ... 简化的示意图
        \end{tikzpicture}
    };

\end{tikzpicture}

\end{document}