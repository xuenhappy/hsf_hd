\documentclass[tikz,border=10pt]{standalone}
\usepackage{tikz}
\usepackage{tikz-3dplot}
\usepackage{amsmath}
\usetikzlibrary{decorations.pathmorphing, arrows.meta, calc, fadings}

% --- 颜色定义 (仿照原图风格) ---
\definecolor{atomRed}{RGB}{205, 70, 60}       % 更加扁平的深红色
\definecolor{atomBorder}{RGB}{140, 40, 30}    % 球体边缘颜色
\definecolor{spinPurple}{RGB}{110, 40, 160}   % 自旋紫
\definecolor{cloudBlue}{RGB}{235, 248, 252}   % 背景淡蓝
\definecolor{waveGreen}{RGB}{80, 180, 100}    % 纠缠绿
\definecolor{textDark}{RGB}{60, 60, 60}       % 字体黑灰

\begin{document}

% 设置3D视角
\tdplotsetmaincoords{65}{115}

\begin{tikzpicture}[
    tdplot_main_coords,
    scale=1.0,
    % --- 样式 ---
    gridline/.style={black!80, thick},
    % 扁平化原子样式: 纯色填充 + 描边
    atom/.style={fill=atomRed, draw=atomBorder, thick, opacity=1.0},
    % 自旋样式
    spin/.style={-{Latex[length=3mm, width=2mm]}, spinPurple, ultra thick},
    % 波浪样式
    wave/.style={waveGreen, thick, decorate, decoration={snake, amplitude=1.2mm, segment length=3.5mm, pre length=1mm, post length=1mm}},
    % 标签文本样式
    label text/.style={font=\bfseries\large, align=center}
]

    % 定义网格参数
    \def\spacing{1.8} 
    \def\gridsize{3} % 4x4 grid (0 to 3)

    % ==========================================
    % 1. 背景云团 (位置稍作调整以适应视角)
    % ==========================================
    \begin{scope}[canvas is xy plane at z=-0.6]
        \fill[cloudBlue, rounded corners=40pt] 
            (-2, -2) -- (\gridsize*\spacing + 2, -1.5) -- 
            (\gridsize*\spacing + 2.5, \gridsize*\spacing + 2.5) -- 
            (-1.5, \gridsize*\spacing + 2) -- cycle;
    \end{scope}

    % ==========================================
    % 2. 绘制网格 (Form 的骨架)
    % ==========================================
    \foreach \i in {0,...,\gridsize} {
        \draw[gridline] (\i*\spacing, 0, 0) -- (\i*\spacing, \gridsize*\spacing, 0);
        \draw[gridline] (0, \i*\spacing, 0) -- (\gridsize*\spacing, \i*\spacing, 0);
    }

    % ==========================================
    % 3. 绘制相互作用波 (纠缠)
    % ==========================================
    % 放在球体后面绘制,或者部分遮挡
    \draw[wave] (0*\spacing, 1*\spacing, 0) -- (1*\spacing, 2*\spacing, 0);
    \draw[wave] (2*\spacing, 0*\spacing, 0) -- (3*\spacing, 1*\spacing, 0);
    \draw[wave] (1*\spacing, 1*\spacing, 0) -- (2*\spacing, 1*\spacing, 0);
    \draw[wave] (2*\spacing, 2*\spacing, 0) -- (2*\spacing, 3*\spacing, 0);

    % ==========================================
    % 4. 绘制电子云 (流) - 简单的箭头表示
    % ==========================================
    % 简化为扁平的小箭头,风格统一
    \newcommand{\drawElectron}[3]{
        \begin{scope}[shift={(#1*\spacing,#2*\spacing,0.2)}, rotate around z=#3]
             % 扁平化流线
             \fill[cyan!30, opacity=0.7] (0,0) to[out=90,in=180] (0.4,0.15) to[out=0,in=90] (0.8,0) 
                          to[out=-90,in=0] (0.4,-0.15) to[out=180,in=-90] (0,0);
             \draw[->, cyan!60!blue, thick] (0.1,0) -- (0.7,0);
        \end{scope}
    }
    \drawElectron{0.5}{0.5}{45}
    \drawElectron{2.5}{1.5}{-10}

    % ==========================================
    % 5. 绘制原子与自旋 (扁平风格)
    % ==========================================
    \foreach \x in {0,...,\gridsize} {
        \foreach \y in {0,...,\gridsize} {
            % 坐标
            \pgfmathsetmacro{\px}{\x*\spacing}
            \pgfmathsetmacro{\py}{\y*\spacing}
            
            % 绘制红球 (直接画圆,不用 shading,实现扁平化)
            % 注意:在 TikZ 3D 坐标中画 circle 会自动投影,产生椭圆效果,符合透视
            % 为了让球看起来是正圆(像图标一样),我们可以在节点中画,或者接受轻微透视
            % 这里使用 shift 坐标画圆,保持一定的 3D 空间感但颜色是平的
            \filldraw[atom] (\px, \py, 0) circle (0.4cm);
            
            % 绘制自旋
            \draw[spin] (\px, \py, 0.25) -- ++(0, 0, 1.2);
        }
    }
    
    % 特殊倾斜自旋
    \draw[spin, rotate around={25:(1*\spacing,1*\spacing,0)}] (1*\spacing, 1*\spacing, 0.25) -- ++(0, 0, 1.2);

    % ==========================================
    % 6. 紧凑的低空注释
    % ==========================================
    
    % --- Form 注释 ---
    % 起点:左上方的网格线交叉点附近
    % 终点:文字标签 (高度降低到 2.0 左右)
    \coordinate (FormTarget) at (0.2, 0.2, 0); 
    \coordinate (FormLabel) at (-1.0, -1.0, 2.0);
    
    \draw[textDark, thick] (FormTarget) -- (FormLabel) 
        node[label text, above, anchor=south east] {Form};
        
    % --- Qualia 注释 ---
    % 起点:右侧的一个自旋箭头尖端
    % 终点:文字标签 (高度降低,距离拉近)
    \coordinate (QualiaTarget) at (\gridsize*\spacing, \gridsize*\spacing, 1.45);
    \coordinate (QualiaLabel) at (\gridsize*\spacing + 0.5, \gridsize*\spacing + 0.5, 2.5);
    
    \draw[spinPurple, thick] (QualiaTarget) -- (QualiaLabel) 
        node[label text, above, text=spinPurple, anchor=south west] {Qualia};

\end{tikzpicture}

\end{document}